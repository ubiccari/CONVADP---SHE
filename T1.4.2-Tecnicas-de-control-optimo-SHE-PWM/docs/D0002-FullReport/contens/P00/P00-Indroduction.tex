\chapter{Introduction}

%
En este documento se propone una solución del problema Selective Harmonic Elimination (SHE) desde el punto de vista de la teoría del control. 
%
El problema SHE consiste en la búsqueda de una función $\{ f(\tau ) \ | \ \tau \in [0,2\pi] \}$ dados unos pocos coeficientes de Fourier. Además esta función $f(\tau)$ solo podrá tomar valores en una discretización del intervalo $[-1,1]$. Nos centraremos en concreto en las funciónes $f(\tau)$ con simetría de media onda, es decir funciones tal que $f(\tau) = -f(\tau + \pi)$, por lo que la descripción de la función $f(\tau)$ queda determinada con sus valores en el intervalo $\tau \in [0,\pi]$. De esta forma, nos referiremos a una función $\{ f(\tau)  | \tau \in [0,\pi] \}$ cuyo desarrollo en serie de Fourier se puede escribir como:

\begin{gather}
    f(\tau ) = \sum_{i \in odd} a_i \cos(i\tau)+ \sum_{j \in odd}  b_j \sin(j \tau) 
\end{gather}

Donde $a_i$ y $b_j$  son:
\begin{gather}
    a_i = \frac{2}{\pi} \int_0^\pi f(\tau ) \cos(i \tau)d\tau \ | \ \forall i \ odd \label{an}\\
    b_j = \frac{2}{\pi} \int_0^\pi f(\tau)  \sin(j \tau) d\tau \ | \ \forall j \ odd \label{bn}
\end{gather}


Entonces el problema de SHE tomando solo los puntos $\{-1,1\}$ como discretización del intervalo $[-1,1]$, se puede formular de la siguiente manera:

\begin{problem}[SHE para dos niveles]\label{SHEp}
    Dado dos conjuntos de números impares $\mathcal{E}_a$ y $\mathcal{E}_b$ con cardinalidades $|\mathcal{E}_a| = N_a$ y  $|\mathcal{E}_b| = N_b$ respectivamente, y dado los vectores objetivo $\bm{a}_T  \in \mathbb{R}^{N_a}$ y $\bm{b}_T  \in \mathbb{R}^{N_b}$, buscamos una función  $f(\tau ) \ | \ \tau \in (0,\pi)$ tal que $f(\tau)$ solo pueda tomar los valores  $\{-1,1\}$ y cuyos coeficientes de Fourier satisfagan: $ a_i = (\bm{a}_T)_i \ | \ \forall i \in \mathcal{E}_a$ y  $b_j = (\bm{b}_T)_j \ \forall \ | \  j \in \mathcal{E}_b$. 
\end{problem}


En la formulación típica de este problema se considera la búsqueda de las posiciones donde ocurren los cambios de la función, estas localizaciones son llamados ángulos de conmutación. Dado que la función $f(\tau)$ solo puede tomar los valores $\{-1,1\}$ los ángulos de conmutación determinan completamente la forma de $f(\tau)$.
%
Por otra parte, dado los vectores $\bm{a}^T$ y $\bm{b}^T$, el número de ángulos $M$ es \emph{a priori} desconocida, por lo que es necesario prefijarla sin un criterio muy claro.
%
Si denotamos ángulos de conmutación como $\bm{\phi} = [\phi_1,\phi_2,\dots,\phi_M] \in \mathbb{R}^M $,  podemos escribir los coeficientes de Fourier (\ref{an}) y (\ref{bn}) en función de $\bm{\phi}$ como: 
\begin{gather}
    a_i(\bm{\phi})  = \dots  \ | \ \forall i \in \mathcal{E}_a \\
    b_j(\bm{\phi})  =  \frac{4}{j\pi  } \bigg[ -1 + 2\sum_{k=1}^M  (-1)^{k+1}\cos(n\phi_k) \bigg] \ | \ \forall j \in \mathcal{E}_b
\end{gather}

De esta manera, dada la expresión anterior podemos formular el problema (\ref{SHEp}) como el siguiente problema de minimización:

\begin{problem}[Minimización para SHE de dos niveles]\label{SHEp_clas}
    Dado dos conjuntos de números impares $\mathcal{E}_a$ y $\mathcal{E}_b$ con cardinalidades $|\mathcal{E}_a| = N_a$ y  $|\mathcal{E}_b| = N_b$,  dado los vectores $\bm{a}_T  \in \mathbb{R}^{N_a}$ y $\bm{b}_T  \in \mathbb{R}^{N_b}$ respectivamente, y dado el número de conmutaciones $M$, buscamos las localizaciones de los ángulos de conmutación $\bm{\phi} \in \mathbb{R}^M$ mediante la siguiente minimización:
    \begin{gather}
        \min_{\bm{\phi} \in \mathbb{R}^M} \Big[
        \sum_{i \in \mathcal{E}_a} (a_i(\bm{\phi}) - a^i_T)^2 + 
        \sum_{j \in \mathcal{E}_b} (b_j(\bm{\phi}) - b^j_T)^2  
        \Big] \\
        \text{sujeto a:} \\ 
            0 < \phi_1  < \phi_2 < \dots  < \phi_{M-1}  <   \phi_{M} < \pi
        \label{constraints}
    \end{gather} 
    Donde las restricciones (\ref{constraints})  conservan el orden de los ángulos, de manera que $\phi_k$ hace referencia al ángulo de conmutación $k$-ésimo. 
    
\end{problem}

A continuación numeraremos las ventajas y desventajas de esta formulación:


\begin{enumerate}
    \item \textbf{Ventajas} 
    \begin{enumerate}
        \item Dado que conocemos que la forma de la función $f(\tau)$ solo dos valores podemos reducir la representación de la variable de optimización a unos pocos ángulos de conmutación. En caso contrario, $f(\tau)$ debería ser representado por su valor dentro del intervalo $[0,\pi]$.
        \item Es un problema NLP (Nonlinear Programming) cuya solución es conocida para unos vectores $\bm{a}_T$ y $\bm{b}_T$ fijos.
    \end{enumerate}
    \item \textbf{Desventajas}
    \begin{enumerate}
        \item El número de ángulos de conmutación $M$ no es conocido para vectores objetivos $\bm{a}_T$ y $\bm{b}_T$.
        \item No se tiene una expresión explícita del problema que pueda ser utilizado a tiempo real para cualquier valor de los targets $\bm{a}_T$ y $\bm{b}_T$.
        \item Existen discontinuidades de la soluciones con respecto a una variación continuada de los vectores de $\bm{a}_T$ y $\bm{b}_T$. Sin embargo esto puede ser intrínseco del problema y no provocado por la formulación.
    \end{enumerate} 
\end{enumerate}


\section{Propuesta y motivación}

Inspirados en la naturaleza continua de la variable de optimización $f(\tau)$, proponemos en este documento la formulación desde el control óptimo. Es decir en lugar buscar los ángulos de conmutación $\bm{\phi} \in \mathbb{R}^M$, buscaremos la función $f(\tau) \in \{ g(\tau)  \in L^\infty([0,\pi])\ /\ |g(\tau)| < 1\} $ que tenga los coeficientes de Fourier deseados. Si utilizamos el  teorema fundamental del cálculo diferencial podemos re-escribir la expresión de los coeficientes de Fourier (\ref{an}) y (\ref{bn}) como la evolución de sistemas dinámicos. Es decir:

\begin{gather}
    \alpha_i(\tau) = \frac{2}{\pi}\int_0^\tau f(\tau) \sin(i\tau)d\tau 
    \Rightarrow
    \begin{cases} \label{ode}
        \dot{\alpha_i}(\tau) & = \frac{2}{\pi}f(\tau)\sin(i\tau) \\  
        \alpha_i(0) & = 0       
    \end{cases}
\end{gather}

\begin{gather}
    \beta_j(\tau) = \frac{2}{\pi}\int_0^\tau f(\tau) \cos(j\tau)d\tau 
    \Rightarrow
    \begin{cases} \label{ode}
        \dot{\beta}_j(\tau) & = \frac{2}{\pi}f(\tau)\cos(j\tau) \\  
        \beta_j(0) & = 0       
    \end{cases}
\end{gather}

La evolución de los sistemas dinámicos $\alpha_i(\tau)$ y $\beta_j(\tau)$ desde el tiempo $\tau=0$ hasta $\tau=\pi$ nos da lugar a los coeficientes $a_i$ y $b_j$. 
De esta manera, el problema general de SHE (\ref{SHEp}) puede resolverse como un problema de control de un sistema dinámico donde $\alpha_i(\tau) \ | \ \forall i \in \mathcal{E}_a  $ y $ \beta_j(\tau) \ \forall j \in \mathcal{E}_b$ son los estados del sistema y $f(\tau)$ la variable de control, y cuyo objetivo será llevar los estados desde el origen de coordenadas hasta los vectores objetivos $\bm{a}_T$ y $\bm{b}_T$ en tiempo $\tau = \pi$
\newline

Este documento esta estructurado de la siguiente manera:
\begin{enumerate}
    \item Capítulo 1. \emph{Open-loop Optimal Control}
    \item Capítulo 2. \emph{Feedback Optimal Control}
\end{enumerate}