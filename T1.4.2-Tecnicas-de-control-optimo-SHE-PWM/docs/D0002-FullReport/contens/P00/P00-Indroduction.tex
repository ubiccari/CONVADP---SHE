\chapter{Introduction}

In this document, we propose  a optimal control  perspective of selective harmonic elimination problem (SHE) with symmetry of half cycle. 
%
In mathematical point of view, SHE problem can be seen as search of a square wave function  $f(\tau ) \ | \ \tau \in (0,2\pi)$ which have fixed a few Fourier coefficients. 
\newline
%
In this way, the $f(\tau)$ can be written in Fourier series as follows:

\begin{gather}
    f(\tau ) = \sum_{i \in odd} a_n \cos(i\tau)+ \sum_{j \in odd}  b_n \sin(j \tau) 
\end{gather}

Where $a_n$ and $b_n$ coefficients are:
\begin{gather}
    a_i = \frac{2}{\pi} \int_0^\pi f(\tau ) \cos(i \tau)d\tau \ | \ \forall i \ odd \label{an}\\
    b_j = \frac{2}{\pi} \int_0^\pi f(\tau)  \sin(j \tau) d\tau \ | \ \forall j \ odd \label{bn}
\end{gather}


Donde se ha considerado que $f(\tau)$ tiene simetría de media onda por lo que  los coeficientes de Fourier $a_n$ y $b_n$ solo son distintos de cero solo cuando $n$ es impar. Entonces el problema de selective harmonic eliminaton se puede formular de la siguiente manera:

\begin{problem}[General SHE for two levels]\label{SHEp}
    Given  a two set of odd numbers $\mathcal{E}_a$ and $\mathcal{E}_b$ with carinalities $|\mathcal{E}_a| = N_a$ and  $|\mathcal{E}_b| = N_b$, and given the target vectors $\bm{a}_T  \in \mathbb{R}^{N_a}$ and $\bm{b}_T  \in \mathbb{R}^{N_b}$, we search a wave form $f(\tau ) \ | \ \tau \in (0,\pi)$ such that $f(\tau)$ only can take values  $\{-1,1\}$ and its Fourier coefficients satisfies: $ a_i = (\bm{a}_T)_i \ | \ \forall i \in \mathcal{E}_a$ and  $b_j = (\bm{b}_T)_j \ \forall \ | \  j \in \mathcal{E}_b$. 
\end{problem}



In the typical formulation of this problem, the function $f(\tau)$ can be represented by locations  where the function $f(\tau)$ changes its value, this locations are named switching angles.
%
Given a some vectors $\bm{a}^T$ and $\bm{b}^T$, the number of switching angles $M$ is \emph{a priori} unknown, so it's necessary fixed it. If we name switching angles as $\bm{\phi} = [\phi_1,\phi_2,\dots,\phi_M] \in \mathbb{R}^M $, we can simplify the Fourier coefficients as follows:

\begin{gather}
    a_i(\bm{\phi})  = \dots  \ | \ \forall i \in \mathcal{E}_a \\
    b_j(\bm{\phi})  =  \frac{4}{j\pi  } \bigg[ -1 + 2\sum_{k=1}^M  (-1)^{k+1}\cos(n\phi_k) \bigg] \ | \ \forall j \in \mathcal{E}_b
\end{gather}

With this expression, we can formulate the problem (\ref{SHEp}) as the next minimization problem:

\begin{problem}[Minimization problem for SHE]\label{SHEp_clas}
    Given  a two set of odd numbers $\mathcal{E}_a$ and $\mathcal{E}_b$ with carinalities $|\mathcal{E}_a| = N_a$ and  $|\mathcal{E}_b| = N_b$, given the vectors $\bm{a}_T  \in \mathbb{R}^{N_a}$ and $\bm{b}_T  \in \mathbb{R}^{N_b}$, and given a number of switching angles, we search the switching angles $\bm{\phi} \in \mathbb{R}^M$ such that
    \begin{gather}
        \min_{\bm{\phi} \in \mathbb{R}^M} \Big[
        \sum_{i \in \mathcal{E}_a} (a_i(\bm{\phi}) - a^i_T)^2 + 
        \sum_{j \in \mathcal{E}_b} (b_j(\bm{\phi}) - b^j_T)^2  
        \Big] \\
        \text{subject to:} \begin{cases}
            0 < \phi_1  \\
            \phi_k < \phi_{k+1} &  \forall k \in \{1,2,\dots,M \}\\
            \phi_{M} < \pi
        \end{cases} \label{constraints}
    \end{gather} 
    Donde las restricciones (\ref{constraints})  conservan el orden de los ángulos. 
    
\end{problem}


\begin{enumerate}
    \item This formulation don't give a clearly procedure to choose a number of angles $M$.
    \item No se tiene una expresión explícita del problema que pueda ser utilizado a tiempo real para cualquier valor de los targets $\bm{a}_T$ y $\bm{b}_T$.
    \item Existen discontinuidades de la soluciones con respecto a una variación continuada de los vectores de $\bm{a}_T$ y $\bm{b}_T$.
\end{enumerate}



\section{Our contributions}
We propose consider a search of a function $f(\tau)$ directly. In this way, instead of looking for the switching angles $\bm{\phi} \in \mathbb{R}^M$, we look for a function $f(\tau) \in \{ g(\tau)  \in L^\infty([0,\pi/2])\ /\ |g(\tau)| < 1\} $. Thanks to fundamental calculus theorem, we can say:

\begin{gather}
    \alpha_i(\tau) = \frac{2}{\pi}\int_0^\tau f(\tau) \sin(i\tau)d\tau 
    \Rightarrow
    \begin{cases} \label{ode}
        \dot{\alpha_i}(\tau) & = \frac{2}{\pi}f(\tau)\sin(i\tau) \\  
        \alpha_i(0) & = 0       
    \end{cases}
\end{gather}

\begin{gather}
    \beta_j(\tau) = \frac{2}{\pi}\int_0^\tau f(\tau) \cos(j\tau)d\tau 
    \Rightarrow
    \begin{cases} \label{ode}
        \dot{\beta}_j(\tau) & = \frac{2}{\pi}f(\tau)\cos(j\tau) \\  
        \beta_j(0) & = 0       
    \end{cases}
\end{gather}


% Cuando resolvemos la ecuación diferencial ordinaria (\ref{ode}) hasta tiempo $T = \pi/2$ obtenemos el valor del coeficiente de Fourier $b_n$.

% Este se puede ver como una ecuación diferencial ordinaria controlada donde los estados del sistema son $\beta_n$ y el control es $f(\tau) \ | \ \tau \in [0,\pi/2]$.
%
So, this ODE can see as control system, where $\alpha_i(\tau) \ | \ \forall i \in \mathcal{E}_a  $ and $ \beta_j(\tau) \ \forall j \in \mathcal{E}_b$ is the states and $f(\tau)$ is the control variable.
%Entonces se puede plantear el problema de control cuyo objetivo es llevar el sistema $\beta_n(\tau)$ desde el estado nulo hasta $b_T^n$ para cada $n \in \{1,3,5,\dots,N/2 \}$ en un tiempo final $\tau_f = \pi/2$
In this way, the problem (\ref{SHEp}) can be solve via optimal control problem, where the objective search a function $f(\tau)$ which drive the system from coordinates origin to the target vectors $\{ \bm{a}_T,\bm{b}_T\}$. 